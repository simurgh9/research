\documentclass{homework}
\author{Tashfeen, Ahmad}
\class{Doctoral Research}
\date{16 June 2022}
\title{Tashfeen's Research Notes}
\address{University of Oklahoma}

\begin{document} \maketitle

\question What is an efficient way to store prime numbers?

\section{32-bit Prime Numbers}

The largest prime that can be saved using 32 bits is,
\[
    p_i = 4294967291\quad\text{where}\quad i=203280221
\]
Note that $32-\lg(p_i) < 0.000000002$. Therefore, taking 32 bits to save each of the $\pi(2^{32})$\footnote{$\pi(2^{32})$ is the \textit{prime-counting function}: counting the number of primes less than or equal to $2^{32}$.} prime numbers, how much memory do we need? We know that,
\[
    \pi(x) > \frac{x}{\ln(x)} > \frac{x}{\lg(x)}
\]
Then let $x=2^n$,
\[
    \frac{x}{\lg(x)} = \frac{2^n}{n} = \frac{2^n}{2^{\lg(n)}} = 2^{n-\lg(n)}
\]
Taking this rough lower bound, there are at least $2^{32-5}=2^{27}$ prime numbers we need to store. If we take 32 bits each,
\[
    \frac{2^{27}\times 32\text{ bits}}{2^{20} \text{ megabits}} =
    2^{27+5-20} = 2^{12} \text{ megabits} = \frac{2^{12} \text{ megabits}}{8 \text{ bits}} = 2^{12-3}=2^9 = 512 \text{ megabytes}.
\]

Can we do better? Yes, we can! We can write each odd prime number as a sum of prime-gaps to it. Take 7 for example,
\[
    7 = p_4 = (p_2-p_1) + (p_3-p_2) + (p_4-p_3) + (p_5-p_4)
      = (3-2) + (5-3) + (7 - 5) + (11 - 7)
\]
In general,
\[
    p_j = \sum_{k=1}^{k=j}(p_{k+1}-p_k) \quad \Ra \quad p_{j+1} = p_j + (p_{j+1}-p_j)
\]
Therefore, $7 = 5 + 2$. The largest prime-gap between any consecutive primes $p_j$ and $p_{j+1}$ \ie $p_{j-1}-p_{j}$ for $p_j < p_{j+1} \leq p_i = 4294967291$ is known to be 336. If we half all the prime-gaps starting at $p_3-p_2 = 5-3$ then we always get an integer. Hence, we may half 336 to 168 obtaining a number that fits in just 8 bits or a byte!

Instead of saving 32-bit prime numbers themselves, we can save halved distances between them starting at $p_3-p_2 = 5-3$ and ending in $p_{203280221}-p_{203280220}$. Let's perform the above calculation again but this time taking only 8 bits.
\[
    \frac{2^{27}\times 8\text{ bits}}{2^{20} \text{ megabits}} =
    2^{27+3-20} = 2^{10} \text{ megabits} = \frac{2^{10} \text{ megabits}}{8 \text{ bits}} = 2^{10-3}=2^7 = 128 \text{ megabytes}.
\]

Not bad. But this is only a lower bound. Since we know that $\pi(2^{32})=i=203280221$, we can calculate the exact amount of storage.

\[
    (203280221-2) \times 2^{3-20-3} = 193.86312389373779296875 < 194\text{ megabytes}
\]

\section{64-bit Prime Numbers}

Since most machines these days are 64 bit, it makes sense to ask how much memory do we need to save all 64-bit primes. We utilise the same halved-gaps optimisation from the previous section. The largest prime-gap between two consecutive primes which are possible to store using 64-bits is known to be 1550. Half of it is 775 which fits 16 bits or 2 bytes. Let's use the same lower-bound on $\pi(2^{64})$,
\[
    \frac{2^{64-6}\times 16\text{ bits}}{2^{50} \text{ petabits}} =
    2^{58+4-50} = 2^{12} \text{ petabits} = \frac{2^{12} \text{ petabits}}{8 \text{ bits}} = 2^{12-3}=2^{9} = 512 \text{ petabytes}.
\]
Which is a bit much.

\section{8-bit Addressing}

One may wonder in case of 64-bit primes, if the largest half-gap we need to save is 775 \ie
\[
    \floor{\lg(775)}+1=10\text{ bits.}
\]
Then why are we using 2 bytes or 6 extra bits for each gap? This is because the smallest unit of addressable memory is a byte. Even if the operating system needs to save a single bit (say 1), it has to save it as a byte: \texttt{0000 0001}. Due to this, 10 bits can only be saved using 2 bytes or 16 bits. This situation is usually remedied with fancy bit-packing algorithms. Albeit, I am keeping it simple.

\end{document}
